\documentclass[a4paper,9pt]{article}
\input{prmb}

\begin{document}
\begin{table}[h]
\begin{tabular}{ll}
{\textbf{UNIVERSIDAD DEL PAC\'IFICO}}&\\ 
{\textbf{Academic Department of Finance}}&\\ 
{\textbf{Quantitative Finance (1F0111)}}&\\
{ \textbf{Second Semester 2017}}&\\
{ \textbf{Instructor: F. Rosales, TA: J. C\'ardenas}}&\\
\end{tabular}
\end{table}

\noindent\rule{16cm}{0.5pt}
~\\
\begin{center}
{\huge Assessment 5}
\end{center}
\vspace{0.5cm}

\textit{Instructions: 
\begin{enumerate}
\item Select one (and only one) of the following questions and answer it thoroughly. 
\item To submit your answer you must commit a folder with your student ID to \\ \url{https://github.com/franciscorosales-marticorena/QuantFin/tree/master/Sol}. 
\item The deadline to commit your work is  14.11.2017 at 11:30, i.e. you have 24 hours.
\item If you would like to add text to further explain your results, you can add a \texttt{Read Me} file. 
\item Note that the information of your \texttt{commit} is public, this means we can all see your output.  
\item Auxiliary material has been uploaded at \\ \url{https://github.com/franciscorosales-marticorena/QuantFin/tree/master/Aux}. 
\end{enumerate}
}
\vspace{0.5cm}

\begin{enumerate}

	%%%%%%%%%%%%%%%%%%%%%%%%%%%%%%%%%%%%%%%%%%%%%%
	%Q 
	%%%%%%%%%%%%%%%%%%%%%%%%%%%%%%%%%%%%%%%%%%%%%%
	\item \textbf{Cryptocurrencies} \hfill (20pts)\\
	~\\
	In class we learned how to create the cryptocurrency CiupCoin. Now we ask you to customize some aspects of it to make it your own. Namely, you must change
	\begin{enumerate}
	\item   the name of the currency from  CiupCoin to a name of your choice, e.g. XCoin. 
	\item   the reward from 10 to 13 XCoins. 
	\item  the number of confirmations per block from 3 to 5. 
	\end{enumerate}
	To verify that you were successful, take screenshots of your results after executing
	\begin{enumerate}
	\item  \begin{verbatim}make -f makefile.unix &\end{verbatim} 
	\item  \begin{verbatim}./XCoin &\end{verbatim}
	\item  \texttt{./XCoin sendtoaddress xyz 1}, where xyz is the public key corresponding to an account in your second virtual machine. 
	\end{enumerate}	
	%%%%%%%%%%%%%%%%%%%%%%%%%%%%%%%%%%%%%%%%%%%%%%
	%Q 
	%%%%%%%%%%%%%%%%%%%%%%%%%%%%%%%%%%%%%%%%%%%%%%
	\item \textbf{Algorithmic Trading} \hfill (20pts)\\
	\begin{enumerate}
	\item Consider the MSFT time series from 1.11.2006 to 25.11.2016 and write an R script to replicate regression-based  trading approach suggested in class. To show that you were successful compare: %at \texttt{Trading Algorithm.ipynb}. 
	\begin{enumerate}
	\item the partial results from running each function
	\item the resulting confusion matrix 
	\end{enumerate}
	If the results are not identical, provide a detailed explanation and illustrate your arguments using computational examples/counter-examples. 
	\item  Solve the dynamic Markowitz portfolio problem:
	\beqn
	\omegavec_{\mbox{\small Q}}=\argmax_\omegavec\left\{\hat\muvec^\top\omegavec-\frac{\gamma}{2}\omegavec^\top\hat\mSigma\omegavec: \onevec^\top\omegavec=1\right\},
	\eeqn
	using the constant conditional correlation (CCC) model for  risky assets:
	\begin{itemize}
	\item EEM: Ishares Msci Emerging Markets
	\item EFA: Ishares Msci Eafe Index Fund
	\item EWJ: iShares MSCI Japan Index Fund
	\item IEF: Ishares Lehman 7 10 Year Treasury Bond Fund
	\item IEV: iShares SAndP Europe 350 Index Fund
	\item IVV: iShares SAndP 500 Index Fund
	\item RWR: DJ Wilshire REIT ETF
	\item SHY: Ishares Lehman 1 3 Year Treasury Bond Fund
	\item TIP: Ishares Lehman Tips Bond Fund
	\item TLT: Ishares Lehman 20 Year Treasury Bond Fund
	\item VTI: Vanguard Total Stock Market ETF
	\end{itemize}
	Report a matrix of weights (11 rows) from 1.11.2007 to 25.11.2016 reporting your results 
	 \item Apply your results in (a) to each individual asset in (b).  Provide ideas on how to use the information from the algorithmic trading methods to improve your Markowitz results.
	\end{enumerate}
\end{enumerate}
\end{document}

